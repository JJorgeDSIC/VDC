\documentclass[]{article}

%Language
\usepackage[utf8]{inputenc}
\inputencoding{latin1}
\usepackage[spanish]{babel}
\usepackage{hyperref}

\usepackage[]{graphicx}

%opening
\title{Parte 2: \\
	Creaci�n de una visualizaci�n sencilla.
}

\author{Jorge Cano, Javier}

\begin{document}

\maketitle

%\begin{abstract}
%
%\end{abstract}

\section{Motivaci�n}

En esta tarea se desea seleccionar un conjunto de datos de acceso abierto para llevar a cabo una visualizaci�n de los mismos. De entre las fuentes proporcionadas, se ha seleccionado...

\begin{itemize}
	\item Variable independiente: Pa�s de origen (Ordenado-Discreto-Nominal).
	\item Variable dependiente: Viajeros entrados (por pa�s de residencia)(Ordenado-Discreto-Cardinal).
	\item Uso de glifos: Geometr�a de los paises.
	\item Variaci�n de la posici�n: Mapeado sobre el mapa pol�tico.
	\item Variaci�n de la saturaci�n del color: N�mero de personas.
\end{itemize} 



\section{Experimentaci�n y resultados}

\section{Conclusi�n}




\end{document}
